<<<<<<< HEAD
%\documentclass[a4paper,doc]{apa6}
\documentclass[a4paper]{article}
% \graphicspath{{/figures/}{./../figures/}}

\usepackage[english]{babel}
\usepackage[utf8x]{inputenc}
\usepackage[colorlinks=true, urlcolor=blue, citecolor=blue, linkcolor=blue]{hyperref}

\usepackage{apacite}
\usepackage{authblk}  % for authors
\usepackage{xcolor}
\definecolor{mypink}{RGB}{255, 230, 255}

\usepackage{nicefrac}
\usepackage{amsmath}
\usepackage{graphicx}
\usepackage{chngcntr} % appendix references
\usepackage[colorinlistoftodos,prependcaption]{todonotes}
\usepackage{pgfplotstable}
\pgfplotstableset{
	fixed zerofill,
	precision=3,
	col sep = comma,
	search path={../tables/}
}
\pgfkeys{/pgf/number format/precision=2, /pgf/number format/fixed}%

\newcommand{\getValue}[3]{%
	\pgfplotstablegetelem{#1}{#2}\of{#3}%
	\pgfmathprintnumber{\pgfplotsretval}%
}
\newcommand{\getCI}[2]{[\getValue{#1}{Lower}{#2}, \getValue{#1}{Upper}{#2}]}

\usepackage{tikz}
\usepackage{tikzscale} % check if used!

\usepackage[most]{tcolorbox}

\newtcolorbox[auto counter]{NewBox2}[2][]{
	colframe=black,colback=white,
	enhanced,
%	breakable,
	title={Box \thetcbcounter: #2},
	colbacktitle=white,
	coltitle = black,
	detach title,
	before upper={\tcbtitle\quad},
	#1
}

\newcommand{\EJ}[1]{\todo[inline, color=green]{  #1 }}
\newcommand{\Q}[1]{\todo[inline, color=yellow]{  #1 }}
\newcommand{\jv}[2]{{\color{red}\st{#1}}{\color{blue}\bf{#2}}}
\newcommand{\DON}[1]{\todo[inline, color=white]{Don: #1}}
\newcommand{\DONside}[1]{\todo[color=white]{#1}}
%\newcommand{\DONTODO}[1]{{\color{red}{#1}} \addcontentsline{tdo}{todo}{#1}}
\newcommand{\J}[1]{\todo[inline, color=mypink]{#1}}

\graphicspath{{../figures/}}
\newcommand{\hypo}[1]{\ensuremath{\mathcal{H}_{#1}}}
\newcommand{\model}{\mathcal{M}}
\newcommand{\data}{\mathrm{data}}%\mathcal{D}}
\newcommand{\midd}{\ensuremath{|}}
\newcommand{\cohend}{\ensuremath{d}}

\newcommand{\osflink}{\url{https://osf.io/uq8st/}}

\newcommand{\CamererReplication}{\url{https://mfr.osf.io/render?url=https://osf.io/fg4d3/?action=download\%26mode=render}}
\newcommand{\manyLabsLink}{\url{https://mfr.osf.io/render?url=https://osf.io/xufw4/?action=download\%26mode=render}}


\title{A Cautionary Note on Estimating Effect Sizes}
%\shorttitle{Estimating Effect Size} 
\renewcommand{\thefootnote}{\fnsymbol{footnote}}
\author[1]{Don van den Bergh%
	\thanks{Correspondence concerning this article should be addressed to: Don van den Bergh, University of Amsterdam, Department of Psychological Methods, Nieuwe Achtergracht 129B, 1018VZ Amsterdam, The Netherlands. E-Mail should be sent to: donvdbergh@hotmail.com.
}}
\author[1]{Julia M. Haaf}
\author[1,2]{Alexander Ly}
\author[3]{\authorcr Jeffrey N. Rouder} % putt Jeff on a newline to avoid a newline after his first name
\author[1]{Eric-Jan Wagenmakers}
\affil[1]{University of Amsterdam}
\affil[2]{Centrum Wiskunde \& Informatica}
\affil[3]{University of California Irvine}
\date{}
%\affiliation{~}
\renewcommand*{\thefootnote}{\arabic{footnote}}
%
%\threeauthors{Don van den Bergh and Julia M. Haaf and Alexander Ly and Eric-Jan Wagenmakers}{Alexander Ly}{Jeffrey N. Rouder}
%\threeaffiliations{University of Amsterdam}{Centrum Wiskunde \& Informatica}{University of California Irvine}
%\authornote{Correspondence concerning this article should be addressed to: Don van den Bergh, University of Amsterdam, Department of Psychological Methods, Nieuwe Achtergracht 129B, 1018VZ Amsterdam, The Netherlands. E-Mail should be sent to: donvdbergh@hotmail.com.}


\begin{document}
\pgfplotstableread{effectSizeExample.csv}\tbEffectSizeExample
\pgfplotstableread{posteriorProbH0.csv}\reanalysis

%\listoftodos
%\newpage

\maketitle

\begin{abstract}
	An increasingly popular approach to statistics is to focus on estimation and to forgo hypothesis testing altogether. Through an example, we show that estimates and confidence of effect sizes are overestimated the null is ignored, and when it is a plausible description of the data. We illustrate how this overestimation can be avoided by incorporating the plausibility of the null into the estimation process.
\end{abstract}

Consider the following scenario: Your colleague has just conducted an experiment for a Registered Report. The analysis yields $p<0.05$ and your colleague believes that the null hypothesis can be rejected. In line with recommendations both old \cite<e.g., >{Grant1962, Loftus1996} and new \cite<e.g., >{harrington2019new, Cumming2014} you convince your colleague that it is better to replace the $p$-value with an estimate of the effect size and a 95\% confidence interval \cite<but see>{MoreyEtAl2016CI}. You also manage to convince your colleague to plot the data. Instead of simply reporting $p<0.05$, the statistical analysis in the report is now more informative. The result is shown in Figure~\ref{fig:descriptivesPlot}. In the text of the paper, the result is summarized as Cohen's $\cohend = \getValue{0}{Estimate}{\tbEffectSizeExample}$, CI $= \getCI{0}{\tbEffectSizeExample}$, in line with guidelines for reporting statistics (e.g., the guidelines of the Psychonomic Society\footnote{\protect\url{https://www.springer.com/psychology?SGWID=0-10126-6-1390050-0}}, or those of Psychological Science\footnote{\url{https://www.psychologicalscience.org/publications/psychological\_science/ps-submissions\#STAT}}).
\begin{figure}[!ht]
	\includegraphics[width=\textwidth]{descriptivesPlot.pdf}
	\caption{The left panel shows a descriptives plot with the mean and 95\% confidence interval of the simulated plant growth. The right panel shows an estimate of the effect size, Cohen's \cohend, and a 95\% confidence interval.}
	\label{fig:descriptivesPlot}
\end{figure}
Given the results shown in Figure~\ref{fig:descriptivesPlot}, what is a reasonable point estimate of the effect size? A straightforward answer is ``\getValue{0}{Estimate}{\tbEffectSizeExample}'' which makes intuitive sense from an estimation perspective. However, your colleague now tells you about the nature of the experiment: plants grow faster when you talk to them.\footnote{Specifically, imagine your colleague took 100 plants and measured their growth three times during two weeks. The first week 50 plants were randomly selected and spoken to while the other served as control. The next week, the roles reversed and the previously spoken to plants served as controls while the control plants were now talked to. The quantity of interest is the difference in growth between the weeks. This example is inspired by \protect\cite{BergerDelampady1987}.} Suddenly, a population effect size of ``0'' also appears plausible. Any observed difference may merely be due to the individual differences between the plants.\footnote{Unless you talk out loud, with consumption, and the plant is near.}

\section*{When are Effect Sizes Overestimated?}
Point estimates and confidence intervals are based on the alternative hypothesis and tend to overestimate effect sizes. This overestimation is caused by the strong assumption that the null or perinull hypothesis is irrelevant. However, the null or perinull can have high plausibility after seeing the data, thus, there can be evidence for the absence of an effect. This happens if the alternative is a-priori unlikely (e.g., for a null hypothesis like ``Talking to plants does not affect their growth.''), or when the data are so uninformative that after seeing the data there is substantial uncertainty about which model best describes the data. If the null hypothesis still has a high plausibility after seeing the data, it is obvious that it cannot be ignored, but that is exactly what is done when estimates are only based on the alternative hypothesis. As a result, the estimates are overconfident and, because the null hypothesis would shrink the estimates towards zero, overestimated.

\section*{A Spike-and-Slab Perspective}
Here, we illustrate the overestimation and a remedy against it by reanalyzing the hypothetical data from Figure~\ref{fig:descriptivesPlot}.\footnote{R code for the analysis is available at \osflink{}.} We consider the spike-and-slab model, which consists of two components. The first component corresponds to the position that talking to plants does not make them grow faster or slower ($\cohend = 0$). The second component corresponds to the position that  speaking to plants does influence their growth ($\cohend \neq 0$). Here we view the spike-and-slab as a single model, although it can also be viewed as a form of Bayesian model averaging (see Box~\ref{box:box1} for more details). Typically, an estimate of the effect size is solely based on the second component, which yields a point estimate and an uncertainty interval (for frequentists, $\cohend = \getValue{0}{Estimate}{\tbEffectSizeExample}$, 95\%  CI: \getCI{0}{\tbEffectSizeExample}; for Bayesians $\cohend = \getValue{1}{mean}{\reanalysis}$, 95\% CRI: \getCI{1}{\reanalysis}). The spike-and-slab model also considers the possibility that an effect can be exactly zero and thus the estimate is an average of the two components and thus shrunken towards zero.\footnote{For the spike-and-slab model, the posterior distribution is constructed in the following manner: $p(\delta\midd\data) = 1\{\delta = 0\}pr(\model_0\midd\data) + p(\delta\midd\data,\model_1)pr(\model_1\midd\data)$. Here, $1\{\delta = 0\}$ is the Dirac delta function which represents the spike under the null, $pr$ denotes probability of a model, and $p$ denotes density related to the magnitude of the effect. Posterior model probabilities are obtained using prior model probabilities of \nicefrac{1}{2}.} Figure 2 contrasts inference based on one component with inference based on both components by showing their 95\% credible intervals and posterior means.
\begin{figure}[!ht]
	\centering
	\begin{tikzpicture}
		\node[anchor=south west,inner sep=0] (image) at (0,0) {\includegraphics[width=0.9\textwidth]{spikeAndSlabPosteriorRescaledPosteriorMode.pdf}};
		\begin{scope}[x={(image.south east)},y={(image.north west)}]
		\node[anchor=base,inner sep=0pt, outer sep=0pt] at (0.28,0.61) {$p(\hypo{0}\mid\data) = \getValue{0}{ph0}{\reanalysis}$};
		\end{scope}
	\end{tikzpicture}
	\caption{Visualization of the results from the spike-and-slab model. The black line represents the posterior distribution of effect size given the slab (i.e., the effect is non-zero). The posterior is scaled so that its mode equals the posterior probability of the alternative model. The gray line represents the posterior probability of the spike (i.e., the effect is exactly 0). The error bars and dots above the density show 95\% credible intervals and the posterior mean for a posterior based on only the slab and for a posterior based on the spike-and slab.}
	\label{fig:modelAveragedPosterior}
\end{figure}
The posterior mean and 95\% credible interval of the spike-and-slab model are shrunken towards 0 compared to the results based on only the slab (\getValue{0}{mean}{\reanalysis} (95\% CRI: \getCI{0}{\reanalysis}) vs. \getValue{1}{mean}{\reanalysis} (95\% CRI: \getCI{1}{\reanalysis})). This shrinkage is caused by the probability that the effect is absent, i.e., the null hypothesis is non-negligible. Note that although 0 appears to be included in the credible interval, this cannot be interpreted directly as evidence against an effect (or significance testing). Instead, the spike-and-slab posterior contains the evidence for there being an effect, combined with the evidence of the size of the effect across the two components. Note that as the posterior probability of the null decreases, the spike-and-slab results approach those of the alternative model.
%\cite<see also>{HaafEtAl2019Nature}
\section*{Discussion}
We argue that estimates of effect sizes based only on the alternative hypothesis tend to be overconfident, in particular when a null or perinull hypothesis may also describe the data well. Consequently, point estimates and confidence intervals based solely on the alternative overestimate effect sizes. A solution for this overestimation is the spike-and-slab model which explicitly considers the possibility that an effect is exactly 0 \cite<as is also advocated for by>{KiersTendeiro2019, RouderEtAl2018PBR}. Although this idea is not new, the influence of the null is still too often ignored in practice.

This approach may contrast with the popular estimation mindset, where it is argued that statistical significance should be abandoned in favor of estimation \cite{McShane2019abandon, valentine2015life, Cumming2014}. Some may argue that all null hypotheses are false \cite{Cohen1990, Meehl1978} and therefore there is no need to consider the component that states that an effect is exactly 0. However, there are statistical motivations to consider a point null \cite{HaafEtAl2019Nature, BergerDelampady1987} and several large-scale replications studies have demonstrated that a near-zero effect size is reasonable in practice, i.e., when the null is a plausible description of the data \cite<e.g., see the meta-analyses conducted by>{KleinEtAl2018ML2, CamererEtAl2018, NosekLakens2014}.

\subsubsection*{When (not) to consider the spike}
When there is overwhelming evidence that an effect is non-zero, the results from a spike-and-slab model become virtually identical to those of a slab only model. Thus, in some situations, it can be argued that it is justified to ignore the spike and that the use of the spike-and-slab model is needlessly complicated. There are two main reasons for ignoring the spike. First, if the effect is known to be large then the posterior probability of the spike will be negligible. Second, if the presence of the effect is established, for instance by prior research, it is sensible to discard the spike a-priori. Yet, the spike does not hurt in either of these situations.

\subsubsection*{Conclusion}
In sum, we argue that descriptions of effect sizes based on only the alternative hypothesis are overconfident and as a consequence overestimate effect sizes. A remedy for this overestimation is the spike-and-slab model which explicitly accounts for the possibility of the null of no effect. Although this idea is not new, it remains underutilized in practice, and we hope this paper brings more attention to the spike-and-slab approach.

\bibliographystyle{apacite}
\bibliography{referenties, referenties.bib}

\begin{NewBox2}[label=box:box1]{The Spike-and-Slab as Bayesian model averaging}{}   
	%	\textbf{Box 1: The Spike-and-Slab as Bayesian model averaging}
	%	\label{box:box1}
	\vspace{6pt}\hrule\vspace{6pt}
	The spike-and-slab can be seen as a single model that consists of two components: the slab, which accounts for the possibility that the effect is non-zero, and the spike, which accounts for an effect of exactly 0. 
%	We view the spike-and-slab as a single model that consists of two components: the slab, which accounts for the possibility that the effect is non-zero, and the spike, which accounts for an effect of exactly 0. 
	However, the spike-and-slab can also be seen as a form of Bayesian model averaging. From that perspective, the spike and the slab are two individual models. The slab represents the unconstrained model that freely estimates effect size, and the spike represents the constrained model where the effect size is fixed to 0. Next, the results for each model are weighted by the posterior model probabilities and averaged, so that inference can be made using results that are averaged over the models considered. It has been shown repeatedly that averaging over the models considered provides the best predictive performance (\citeNP[pp. 640--641]{ZellnerVandaele1975}, as described in \citeNP[p. 600--601]{ZellnerSiow1980}; \citeNP[p. 57]{Haldane1932}, \citeNP{IversonEtAl2010}, \citeNP{RouderEtAl2018PBR}), and conceptually similar ideas date back much further (\citeNP[p. 387]{WrinchJeffreys1921}, \citeNP{Jevons18741913}). Note that these two perspectives ---a two-component model or averaging of two models--- differ in semantics but are mathematically equivalent. Here we view the spike-and-slab as a single model because we focus on estimation.
\end{NewBox2}

%\clearpage

%\appendix
%\counterwithin{figure}{section}
%\begin{figure}[!ht]
%	\includegraphics[width=\textwidth]{posteriorMeanVsSampleDelta3panel.pdf}
%	\caption{Posterior mean for effect size (y-axis) as a function of the observed effect size (x-axis). The left panel shows inference conditional on the alternative model (i.e., the slab). The middle panel shows the model averaged posterior mean, which shrinks towards 0 as the sample effect size approaches 0 and the null model becomes more plausible. The right panel shows the a 95\% credible interval for the model averaged posterior. The colors and line types represent different variances of the prior distribution.}
%	\label{fig:posteriorMeanVsSampleDelta}
%\end{figure}


% comment graveyard

%With sparse data, the prior distribution shrink the estimate towards its mean, in this case 0.\DON{I know the prior shrinks estimates towards 0, but this has little to do with the true effect size. Rather, it's a reason why Bayesian estimators (of effect size) are be biased and tend to underestimate the true value.}
% In addition, since the null hypothesis is considered a-priori plausible, its predictions should be considered also. It has been shown repeatedly that model averaging provides the best predictive performance (\citeNP[pp. 640-641]{ZellnerVandaele1975}, as described in \citeNP[p. 600-601]{ZellnerSiow1980}; \citeNP[p. 57]{Haldane1932}, \citeNP{IversonEtAl2010}, \citeNP{RouderEtAl2018PBR}), and conceptually similar ideas date back much further (\citeNP[p. 387]{WrinchJeffreys1921}, \citeNP{Jevons18741913}). Thus, to obtain the best predictions one should model average over the null and alternative hypothesis. The impact of the null will shrink the estimate of effect size towards 0.
% 
% \J{We should probably first introduce the two relevant models here, and the idea that an unconstrained model is used for estimation. Maybe even say something like noone would use the null model to generate effect size estimates, why would we use the unconstrained?}

%\DONside{Let d be the effect size of the model averaged posterior predictive distribution. Consider the limit of d as the number of posterior predictive observations goes to infinity. Do we retrieve the model averaged posterior effect size? Intuitively yes, which could be a nice argument that the model averaged effect size can definitely be interpreted for this model.}

%Show prior-posterior plot from JASP:
%With sparse data, the prior distribution on effect size will shrink the estimate towards 0. Happens with the default settings, but even more when the width is smaller.
%
%Also, show spike at zero (maybe we need better JASP plot, will ask Don to create it):
%
%The impact of H0 will shrink the estimate toward 0. This is most clear when H0 is a priori very likely (''plants do not grow faster when you pray for them'') or when the sample effect is very close to zero, so that it becomes clear that H0 might provide a more reasonable explanation.
%
%Mention earlier literature:
%Model averaging effect size:
%\cite[pp. 640-641]{ZellnerVandaele1975}, as described in \cite[p. 600-601]{ZellnerSiow1980}; also \cite[p. 57]{Haldane1932}, \cite{IversonEtAl2010}
%
%Early ideas that are conceptually similar can be found in \cite[p. 387]{WrinchJeffreys1921} (show BMA between $\mathcal{H}_1$ and $\mathcal{H}_0$ -- but for prediction, not estimating effect size!; see also \cite{Jevons18741913}).
%
%See also \cite{RouderEtAl2018PBR}:
%
%Key is Figure 5. The spike-and-slab model shows shrinkage towards zero for small observed effect sizes because the spike has increased influence.
%
%``There are alternative interpretations that we find somewhat cumbersome. One is that the spike-and-slab can be viewed not as a model but as a model-averaging device. Here, the goal is not so much to define categories of effect and no-effect, but to average across both of them, and this averaging results in regularization. If one uses this interpretation, the prior odds settings are important as they influence the posterior weight given to each model component in the averaging. Another alternative interpretation comes from Kruschke and Liddell (Kruschke, 2011; Kruschke \& Liddell, 2017; this issue). Here, the spike and slab are seen as separate components in a hierarchical model. Accordingly, a focus on Bayes factors denotes a focus on the choice between components; a focus on posterior estimation entails parameter estimation after choosing the slab. We find this view difficult inasmuch as there is no a priori reason to choose the slab to focus on estimation. If one admits the possibility of the spike, then assuredly it should affect posterior estimation as well.''

%There are at least two reasons for this tendency to overestimate. The first reason is that there is a high base-rate of null effects which leads to high prior odds in favor of a null effect. The second reason is that when the data are not very overwhelming, the posterior plausibility of a point null can be large. This implies that the null should not be ignored, but this is precisely what happens when inference is based only on the alternative. These reasons originate from the unheeded assumption that a null or perinull hypothesis is irrelevant and lead to overconfident estimates and intervals. Since ignoring the null hypothesis effectively avoids shrinkage towards 0, this overconfidence is associated with an overestimation.

%% Model averaging to ss
%Accordingly, our proposed remedy is to average across the null model and the alternative model, weighted by their posterior model probabilities.\footnote{For effect size, we obtain the following model-averaged posterior distribution: $p(\delta\midd\data) = 1\{\delta = 0\}pr(\model_0\midd\data) + p(\delta\midd\data,\model_1)pr(\model_1\midd\data)$. Here, $1\{\delta = 0\}$ is the Dirac delta function which represents the spike under the null, $pr$ denotes probability of a model, and $p$ denotes density related to the magnitude of the effect. Posterior model probabilities are obtained using prior model probabilities of \nicefrac{1}{2}.} Figure 2 contrasts inference based on the alternative hypothesis only with inference based on the averaged model by showing intervals and posterior means.
%
%\section*{A Bayesian Model-Averaged Perspective}
%Here, we illustrate the overestimation and a remedy against it by reanalyzing the simulated data from Figure~\ref{fig:descriptivesPlot}.\footnote{R code for the analysis is available at \osflink{}.} We consider two hypotheses: The null hypothesis of no effect (\hypo{0}), speaking to plants does not make them grow faster or slower ($\cohend = 0$), and the alternative hypothesis (\hypo{1}), speaking to plants makes them grow faster or slower ($\cohend \neq 0$). We consider both these hypotheses using a paired-samples t-test. Typically, an estimate of the effect size is based on solely the alternative hypothesis, which yields a point estimate and an uncertainty interval (for frequentists, $\cohend = \getValue{0}{Estimate}{\tbEffectSizeExample}$, 95\%  CI: \getCI{0}{\tbEffectSizeExample}; for Bayesians $\cohend = \getValue{1}{mean}{\reanalysis}$, 95\% CRI: \getCI{1}{\reanalysis}). However, it has been shown repeatedly that averaging over the models considered provides the best predictive performance (\citeNP[pp. 640--641]{ZellnerVandaele1975}, as described in \citeNP[p. 600--601]{ZellnerSiow1980}; \citeNP[p. 57]{Haldane1932}, \citeNP{IversonEtAl2010}, \citeNP{RouderEtAl2018PBR}), and conceptually similar ideas date back much further (\citeNP[p. 387]{WrinchJeffreys1921}, \citeNP{Jevons18741913}). Accordingly, our proposed remedy is to average across the null model and the alternative model, weighted by their posterior model probabilities.\footnote{For effect size, we obtain the following model-averaged posterior distribution: $p(\delta\midd\data) = 1\{\delta = 0\}pr(\model_0\midd\data) + p(\delta\midd\data,\model_1)pr(\model_1\midd\data)$. Here, $1\{\delta = 0\}$ is the Dirac delta function which represents the spike under the null, $pr$ denotes probability of a model, and $p$ denotes density related to the magnitude of the effect. Posterior model probabilities are obtained using prior model probabilities of \nicefrac{1}{2}.} Figure 2 contrasts inference based on the alternative hypothesis only with inference based on the averaged model by showing intervals and posterior means.

%The model-averaged posterior mean and credible interval are shrunken towards 0 compared to the posterior mean conditional on the alternative hypothesis (\getValue{0}{mean}{\reanalysis} (95\% CRI: \getCI{0}{\reanalysis}) vs. \getValue{1}{mean}{\reanalysis} (95\% CRI: \getCI{1}{\reanalysis})). This shrinkage is caused by the probability that the effect is absent, i.e., the null hypothesis is non-negligible. Note that although 0 appears to be included in the credible interval, this cannot be interpreted directly as evidence (or significance testing). Instead, the spike-and-slab posterior contains the evidence for there being an effect, combined with the evidence of the size of the effect across the two models. Note that as the posterior probability of the null decreases, the model-averaged results approach those of the alternative model.

%The argument is not affected if the point null is replaced by a perinull.

%A key aspect of model averaging is that it does not require model selection; there is no need to commit to a single model to obtain parameter estimates, although multiple models are considered. Therefore, model-averaged predictions and parameter estimates do fit the philosophy behind focusing on estimation. A skeptic might remark that in order to model average, it is necessary to obtain some form of model evidence and transform this into posterior model probabilities (e.g., Bayes factors, weighted information criteria) which reintroduces the importance of testing. However, the same can be said for estimation based inference, since confidence intervals can be constructed in multiple ways (e.g., bootstrapping, normal approximation), and Bayesian parameter estimates also depend on the choice of the prior distribution. A recent comment argued that uncertainty should be embraced \cite{Amrhein2019scientists}, which is exactly what we are trying to do using model averaging.

%\subsubsection*{When not to model average}
%When the sole aim is prediction rather than estimation and interpretation of parameters, model averaging will likely be beneficial \cite<but see>{grunwald2017inconsistency, minka2000bayesian}. However, if the goal is to interpret or base decisions on parameter estimates, there are several reasons to forgo model averaging. First, if the models are based on competing theories it makes little sense to model average. The model-averaged parameters will be uninterpretable and thus meaningless. Second, the nature of the problem may be ill-suited for model averaging. For example, imagine a new experimental treatment, living at a high altitude, improves patients' quality of life. However, the treatment is only effective if the patients live at a high altitude for at least two years. To encourage patients to complete the treatment, they receive a livelihood subsidy. Here, the goal is to determine how much funding each patient should receive to maximize their quality of life. For each patient, it is unknown how long they will stay, regardless of the funding they receive. Here, one model represents that a patient stays 2 years or more while another model represents that he or she does not. Using some background variables as predictors, we obtain for each patient the posterior probability that they complete the treatment. In this scenario it is meaningless to average the subsidy spent on a patient by the posterior probability of them completing the treatment, as such an average would always provide patients with less than the required amount of funding to complete the treatment, essentially wasting the subsidy. Instead, funding should only be given to patients for which it is likely that they complete the treatment, i.e., some form of model selection is required. More generally, whenever some form of thresholding will be applied to a model's results, model averaging may not be useful \cite<see also>{HaafEtAl2019Nature}.


=======
%\documentclass[a4paper,doc]{apa6}
\documentclass[a4paper]{article}
% \graphicspath{{/figures/}{./../figures/}}

\usepackage[english]{babel}
\usepackage[utf8x]{inputenc}
\usepackage[colorlinks=true, urlcolor=blue, citecolor=blue, linkcolor=blue]{hyperref}

\usepackage{apacite}
\usepackage{authblk}  % for authors
\usepackage{xcolor}
\definecolor{mypink}{RGB}{255, 230, 255}

\usepackage{nicefrac}
\usepackage{amsmath}
\usepackage{graphicx}
\usepackage{chngcntr} % appendix references
\usepackage[colorinlistoftodos,prependcaption]{todonotes}
\usepackage{pgfplotstable}
\pgfplotstableset{
	fixed zerofill,
	precision=3,
	col sep = comma,
	search path={../tables/}
}
\pgfkeys{/pgf/number format/precision=2, /pgf/number format/fixed}%

\newcommand{\getValue}[3]{%
	\pgfplotstablegetelem{#1}{#2}\of{#3}%
	\pgfmathprintnumber{\pgfplotsretval}%
}
\newcommand{\setValue}[4]{%
	\pgfplotstablegetelem{#1}{#2}\of{#3}%
	\pgfmathprintnumberto{\pgfplotsretval}{#4}%
}


\newcommand{\getCI}[2]{[\getValue{#1}{Lower}{#2}, \getValue{#1}{Upper}{#2}]}

\usepackage{titlesec}

\titlespacing\section{0pt}{8pt plus 4pt minus 2pt}{4pt plus 2pt minus 2pt}
\titlespacing\subsection{0pt}{8pt plus 4pt minus 2pt}{0pt plus 2pt minus 2pt}
\titlespacing\subsubsection{0pt}{8pt plus 4pt minus 2pt}{0pt plus 2pt minus 2pt}

\usepackage{tikz}
\usepackage{tikzscale} % check if used!

\usepackage[most]{tcolorbox}

\newtcolorbox[auto counter]{NewBox2}[2][]{
	colframe=black,colback=white,
	enhanced,
%	breakable,
	title={Box \thetcbcounter: #2},
	colbacktitle=white,
	coltitle = black,
	detach title,
	before upper={\tcbtitle\quad},
	#1
}
%\getValue{0}{ph0}{\reanalysis}

\newcommand{\EJ}[1]{\todo[inline, color=green]{  #1 }}
\newcommand{\Q}[1]{\todo[inline, color=yellow]{  #1 }}
\newcommand{\jv}[2]{{\color{red}\st{#1}}{\color{blue}\bf{#2}}}
\newcommand{\DON}[1]{\todo[inline, color=white]{Don: #1}}
\newcommand{\DONside}[1]{\todo[color=white]{#1}}
%\newcommand{\DONTODO}[1]{{\color{red}{#1}} \addcontentsline{tdo}{todo}{#1}}
\newcommand{\J}[1]{\todo[inline, color=mypink]{#1}}

\graphicspath{{../figures/}}
\newcommand{\hypo}[1]{\ensuremath{\mathcal{H}_{#1}}}
\newcommand{\model}{\mathcal{M}}
\newcommand{\data}{\mathrm{data}}%\mathcal{D}}
\newcommand{\midd}{\ensuremath{\,|\,}}
\newcommand{\cohend}{\ensuremath{d}}
\newcommand{\prob}{\mathrm{Pr}}

\newcommand{\osflink}{\url{https://osf.io/uq8st/}}

\newcommand{\CamererReplication}{\url{https://mfr.osf.io/render?url=https://osf.io/fg4d3/?action=download\%26mode=render}}
\newcommand{\manyLabsLink}{\url{https://mfr.osf.io/render?url=https://osf.io/xufw4/?action=download\%26mode=render}}


\title{A Cautionary Note on Estimating Effect Size}
%\shorttitle{Estimating Effect Size} 
\renewcommand{\thefootnote}{\fnsymbol{footnote}}
\author[1]{Don van den Bergh%
	\thanks{Correspondence concerning this article should be addressed to: Don van den Bergh, University of Amsterdam, Department of Psychological Methods, Nieuwe Achtergracht 129B, 1018VZ Amsterdam, The Netherlands. E-Mail should be sent to: donvdbergh@hotmail.com.
}}
\author[1]{Julia M. Haaf}
\author[1,2]{Alexander Ly}
\author[3]{\authorcr Jeffrey N. Rouder} % putt Jeff on a newline to avoid a newline after his first name
\author[1]{Eric-Jan Wagenmakers}
\affil[1]{University of Amsterdam}
\affil[2]{Centrum Wiskunde \& Informatica}
\affil[3]{University of California Irvine}
\date{}
%\affiliation{~}
\renewcommand*{\thefootnote}{\arabic{footnote}}
%
%\threeauthors{Don van den Bergh and Julia M. Haaf and Alexander Ly and Eric-Jan Wagenmakers}{Alexander Ly}{Jeffrey N. Rouder}
%\threeaffiliations{University of Amsterdam}{Centrum Wiskunde \& Informatica}{University of California Irvine}
%\authornote{Correspondence concerning this article should be addressed to: Don van den Bergh, University of Amsterdam, Department of Psychological Methods, Nieuwe Achtergracht 129B, 1018VZ Amsterdam, The Netherlands. E-Mail should be sent to: donvdbergh@hotmail.com.}

\pgfplotstableread{effectSizeExample.csv}\tbEffectSizeExample
\pgfplotstableread{posteriorProbH0.csv}\reanalysis

\begin{document}

%\listoftodos
%\newpage

\maketitle

\begin{abstract}
	An increasingly popular approach to statistical inference is to focus on the estimation of effect size while ignoring the null hypothesis that the effect is absent. We demonstrate how this common ``null hypothesis neglect'' may result in effect size estimates that are overly optimistic. The overestimation can be avoided by incorporating the plausibility of the null hypothesis into the estimation process through a ``spike-and-slab'' model.
\end{abstract}

Consider the following hypothetical scenario: a colleague from the biology department has just conducted an experiment and approaches you for statistical advice. The analysis yields $p<0.05$ and your colleague believes that this is grounds to reject the null hypothesis. In line with recommendations both old \cite<e.g., >{Grant1962, Loftus1996} and new \cite<e.g., >{harrington2019new, Cumming2014} you convince your colleague that it is better to replace the $p$-value with a point estimate of effect size and a 95\% confidence interval \cite<but see>{MoreyEtAl2016CI}. You also manage to convince your colleague to plot the data (see Figure~\ref{fig:descriptivesPlot}). Mindful of the reporting guidelines of the \emph{Psychonomic Society}\footnote{\protect\url{https://www.springer.com/psychology?SGWID=0-10126-6-1390050-0}} and \emph{Psychological Science}\footnote{\url{https://www.psychologicalscience.org/publications/psychological\_science/ps-submissions\#STAT}}, your colleague reports the result as follows: ``Cohen's $\cohend = \getValue{0}{Estimate}{\tbEffectSizeExample}$, CI $= \getCI{0}{\tbEffectSizeExample}$''.

\begin{figure}[!ht]
	\includegraphics[width=\textwidth]{descriptivesPlot.pdf}
	\caption{Standard estimation results for the fictitious plant growth example. Left panel: a descriptives plot with the mean and 95\% confidence interval of plant growth in the two conditions. Right panel: point estimate and 95\% confidence interval for Cohen's \cohend.}
	\label{fig:descriptivesPlot}
\end{figure}

Based on these results, what would be a reasonable point estimate of effect size? A straightforward and intuitive answer is ``\getValue{0}{Estimate}{\tbEffectSizeExample}''. However, your colleague now informs you of the hypothesis that the experiment was designed to assess: ``plants grow faster when you talk to them''.\footnote{Specifically, imagine your colleague selected 100 plants and weighted them three times: at the start of the experiment, after one week, and after two weeks. The first week 50 plants were randomly selected and spoken to, while the others served as controls. The next week the roles were reversed: the previously spoken to plants served as controls while the control plants were now spoken to. The quantity of interest is the difference in weight between the two conditions. This example is inspired by \protect\citeA{BergerDelampady1987}.} Suddenly, a population effect size of ``0'' appears eminently plausible. Any observed difference may merely be due to the inevitable sampling variability.\footnote{Unless your colleague talked out loud, with consumption, and the plants were near.}

\section*{When Are Effect Sizes Overestimated?}
Standard point estimates and confidence intervals ignore the possibility that the effect is spurious (i.e., the null hypothesis $\mathcal{H}_0$). This is not problematic when $\mathcal{H}_0$ is deeply implausible, either because $\mathcal{H}_0$ was highly unlikely \emph{a priori} or because the data decisively undercut $\mathcal{H}_0$. But when the data fail to undercut $\mathcal{H}_0$, or when $\mathcal{H}_0$ is highly likely \emph{a priori} (i.e., ``plants do not grow faster when you talk to them''), then $\mathcal{H}_0$ is not ruled out as a plausible account of the data. Effect size estimates that ignore a plausible $\mathcal{H}_0$ are generally overconfident: the fact that $\mathcal{H}_0$ provides an acceptable account of the data should shrink effect size estimates towards zero.

\section*{A Spike-and-Slab Perspective}
Here we illustrate both the overestimation and a remedy by reanalyzing the fictitious data from Figure~\ref{fig:descriptivesPlot}.\footnote{R code for the analysis is available at \osflink{}.} We apply the spike-and-slab model \cite{RouderEtAl2018PBR, clyde1996prediction, mitchell1988bayesian}, which consists of two components. The first component corresponds to the position that talking to plants does not affect their growth (i.e., $\delta = 0$), whereas the second component corresponds to the position that speaking to plants does affect their growth (i.e., $\delta \neq 0$). Both components are deemed \emph{a priori} equally likely, such that the prior probability for each component is \nicefrac{1}{2}. Here we view the spike-and-slab setup as a single model, although it can also be viewed as a form of Bayesian model averaging (see Box~\ref{box:box1} for details). In almost all current empirical work, an estimate of effect size is based solely on the second component, which yields a point estimate and an uncertainty interval (for frequentists, $\delta = \getValue{0}{Estimate}{\tbEffectSizeExample}$, 95\%  CI: \getCI{0}{\tbEffectSizeExample}; for Bayesians $\delta = \getValue{1}{mean}{\reanalysis}$, 95\% CRI: \getCI{1}{\reanalysis}). The spike-and-slab model, however, also considers the possibility that an effect can be absent; consequently, the overall estimate from the spike-and-slab model is a weighted average of the two components, shrinking the estimate towards zero.\footnote{For the spike-and-slab model, the posterior distribution is constructed in the following manner: $p(\delta\midd\data) = 1\{\delta = 0\}\prob(\model_0\midd\data) + p(\delta\midd\data,\model_1)\prob(\model_1\midd\data)$. Here, $1\{\delta = 0\}$ is the Dirac delta function which represents the spike under $\mathcal{H}_0$, $\prob$ denotes probability of a model, and $p$ denotes density related to the magnitude of the effect.} Figure 2 contrasts the traditional slab-only estimation against the spike-and-slab estimation.
% these cannot be generated inside the caption
\setValue{0}{ph1}{\reanalysis}{\phAlt}%
\setValue{0}{mode}{\reanalysis}{\modeAlt}%
\begin{figure}[!ht]
	\centering
	\begin{tikzpicture}
		\node[anchor=south west,inner sep=0] (image) at (0,0) {\includegraphics[width=0.9\textwidth]{spikeAndSlabPosteriorRescaledPosteriorMode.pdf}};
		\begin{scope}[x={(image.south east)},y={(image.north west)}]
		\node[anchor=base,inner sep=0pt, outer sep=0pt] at (0.28,0.61) {$p(\hypo{0}\mid\data) = \getValue{0}{ph0}{\reanalysis}$};
		\end{scope}
	\end{tikzpicture}
	\caption{
		The spike-and-slab model. The black line represents the posterior distribution of effect size given the slab (i.e., the effect is non-zero). The posterior is scaled so that its mode ($\delta = \modeAlt$) equals the posterior probability of the alternative model (i.e., $p(\hypo{1}\mid\data) = \phAlt$). The grey line represents the posterior probability of the spike (i.e., $\mathcal{H}_0$: the effect is absent). The error bars and dots above the density show 95\% credible intervals and the posterior mean for the slab-only model and for the spike-and-slab model.}
	\label{fig:modelAveragedPosterior}
\end{figure}
Compared to the traditional results based only on the slab, the posterior mean and central 95\% credible interval of the spike-and-slab model are shrunken towards 0 (i.e., \getValue{0}{mean}{\reanalysis} (95\% CRI: \getCI{0}{\reanalysis}) vs. \getValue{1}{mean}{\reanalysis} (95\% CRI: \getCI{1}{\reanalysis})). This shrinkage is due to the non-negligible probability that the effect is absent. The spike-and-slab posterior represents the plausibility that the effect is absent by the height of the spike, and the uncertainty about the effect's magnitude, given that it is present, by the width of the slab. Note that as the posterior probability of $\mathcal{H}_0$ decreases, the spike-and-slab results approach those of the slab-only model.

\section*{Discussion}
Standard estimates of effect size ignore the null hypothesis and are therefore overconfident, that is, biased away from zero. The spike-and-slab model remedies this problem by explicitly considering the possibility that an effect is absent \cite{Robinson2019,RouderEtAl2018PBR}. The core idea dates back to \citeA{Jeffreys1939}; nonetheless, it is ignored both in empirical practice, in statistical education, and in journal guidelines.

%\subsubsection*{Extending the Null to a Perinull}
\subsubsection*{What if All Null Hypotheses Are False?}
The spike-and-slab approach clashes with the popular estimation mindset, where it is argued that statistical significance should be abandoned in favor of estimation \cite{McShane2019abandon, Cumming2016introduction, valentine2015life, Cumming2014}. One argument to forgo hypothesis testing is that all null hypotheses are false \cite{Cohen1990, Meehl1978} and therefore there is no need to consider a component that states that an effect is exactly zero. The statistical counterargument is that, even if point null hypotheses are false, they are still mathematically convenient approximations to more complex perinull hypotheses that allow mass on an interval close to zero \cite{BergerDelampady1987, KiersTendeiro2019}. Thus, from a pragmatic perspective it is irrelevant whether or not null hypotheses are exactly true: in the spike-and-slab model, a perinull ``stake'' or ``chimney'' \cite{KiersTendeiro2019} component will shrink estimates towards zero almost as much as the point null spike component will. 

\subsubsection*{When to Ignore the Spike}
There are two scenarios in which the presence of the spike (or perinull stake) can safely be ignored. First, the spike may be deeply implausible. This happens most often in problems of pure estimation, such as when determining the relative popularity of two politicians or the proportion of Japanese cars on the streets of New York. In such cases, no value or interval needs to be singled out for special attention. Second, the data may provide overwhelming evidence that an effect is present. When this happens, the results from a spike-and-slab model become virtually identical to those of a slab-only model, and the inclusion of the spike does not offer an added benefit. A practical recommendation by Harold Jeffreys is to ignore the spike whenever sample sizes fall between 50 and 2000 and the maximum likelihood estimate deviates from the spike by more than three standard errors (\citeNP[pp. 193--194]{Jeffreys1939}; \citeNP[p. 75]{Jeffreys1980}).

\subsubsection*{Conclusion}
Standard methods for estimating effect size produce results that are overly optimistic. This bias toward high estimates can be corrected by applying the spike-and-slab model which explicitly accounts for the possibility that the effect is absent. 

\newpage
\begin{NewBox2}[label=box:box1]{The Spike-and-Slab Distribution as Bayesian Model Averaging}{}%
	\vspace{6pt}\hrule\vspace{6pt}
	The spike-and-slab distribution can be viewed as a single model that consists of two components: the slab, which assumes that the effect is present, and the spike, which assumes the effect is absent. However, the spike-and-slab distribution can also be seen as a form of Bayesian model averaging. From that perspective, the spike and the slab are two individual models. The slab represents the unconstrained model that freely estimates effect size, and the spike represents the constrained model where the effect size is fixed to zero. Next, the results for each model are weighted by the posterior model probabilities and averaged, so that inference can be made using results from both models simultaneously. Such averaging over models yields optimal predictive performance (\citeNP[p. 640--641]{ZellnerVandaele1975}, as described in \citeNP[p. 600--601]{ZellnerSiow1980}; \citeNP[p. 57]{Haldane1932}; \citeNP{IversonEtAl2010}; \citeNP{RouderEtAl2018PBR}), and conceptually similar ideas date back much further (\citeNP[p. 387]{WrinchJeffreys1921}; \citeNP{Jevons18741913}). Note that these two perspectives ---a two-component model or averaging of two models--- differ in semantics but are mathematically equivalent.
\end{NewBox2}

\newpage

\bibliographystyle{apacite}
% \bibliography{referenties, referenties.bib}
\bibliography{referenties.bib}

>>>>>>> e7d736cc7f84b7bc2a07ba785cbdbc7c3ad021bd
\end{document}